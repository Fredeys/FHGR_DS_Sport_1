% !TEX root = documentation.tex


\titlehead{BSc Computational and Data Science\\CDS Data Scinece\\Dozent: Prof. Corsin Capol\hfill}
\title{Sportdatenanalyse mit dem Hauptfokus auf die Herzfrequenz}
\subtitle{Leistungsnachweis I}
\author[1,*]{Diana Feusi}
\author[1,*]{Michelle Arn}
\author[1,*]{Frederic Kurbel}
\affil[1]{Fachhochschule Graubünden}
\affil[*]{E-Mail Adressen: diana.feusi@stud.fhgr.ch, michelle.arn@stud.fhgr.ch, frederic.kurbel@stud.fhgr.ch}
\date{\today}
\maketitle

\begin{abstract}
In dieser Arbeit wurden die Herzfrequenz und die Trainingsaktivität einer Probandin über den Zeitraum von 2021 bis Mitte 2025 untersucht. Ziel war es, den Einfluss von regelmässiger sportlicher Aktivität auf den Ruhepuls zu analysieren.

Die Ergebnisse zeigen, dass Monate mit höherer Trainingshäufigkeit und längerer Aktivitätsdauer meist mit einem niedrigeren Ruhepuls einhergehen. Der durchschnittliche Ruhepuls blieb über die Jahre weitgehend stabil, wobei leichte Schwankungen beobachtet wurden. Die Aktivitätsdauer und -häufigkeit variierten stärker, was auf unterschiedliche Trainingsintensitäten oder persönliche Lebensumstände zurückgeführt werden kann.

Die Analyse verdeutlicht, dass regelmässiges Ausdauertraining das Herz-Kreislauf-System positiv beeinflussen kann. Aufgrund der Untersuchung einer einzelnen Person sind die Ergebnisse jedoch nur eingeschränkt verallgemeinerbar. Zukünftige Studien sollten weitere Probanden und mögliche Einflussfaktoren wie Schlaf, Stress oder Ernährung berücksichtigen.

Insgesamt liefert die Arbeit Hinweise darauf, dass sportliche Aktivität den Ruhepuls senken und somit die kardiorespiratorische Fitness verbessern kann.
\end{abstract}
